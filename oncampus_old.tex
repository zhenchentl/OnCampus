% Template for ICME-2013 paper; to be used with:
%          spconf.sty  - ICASSP/ICIP LaTeX style file, and
%          IEEEbib.bst - IEEE bibliography style file.
% --------------------------------------------------------------------------
% ������£�
% xie2015mining����ο�����ɾ������δ������
% --------------------------------------------------------------------------
\documentclass{article}
\usepackage{spconf,amsmath,epsfig}

\pagestyle{empty}


\begin{document}\sloppy

% Example definitions.
% --------------------
\def\x{{\mathbf x}}
\def\L{{\cal L}}


% Title.
% ------
\title{A Scheme Towards a Smart Campus}
%
% Single address.
% ---------------
\name{Author(s) Name(s)\thanks{Thanks to XYZ agency for funding.}}
\address{School of Software, Dalian University of Technology, Dalian 116620, China \\
f.xia@ieee.org}
%
% For example:
% ------------
%\address{School\\
%	Department\\
%	Address\\
%   Email}
%
% Two addresses (uncomment and modify for two-address case).
% ----------------------------------------------------------
%\twoauthors
%  {A. Author-one, B. Author-two\sthanks{Thanks to XYZ agency for funding.}}
%	{School A-B\\
%	Department A-B\\
%	Address A-B}
%  {C. Author-three, D. Author-four\sthanks{The fourth author performed the work
%	while at ...}}
%	{School C-D\\
%	Department C-D\\
%	Address C-D\\
%   Email}
%

\maketitle


%
\begin{abstract}
  Recent years, there are increasing number of researchers working on the constructions of smart city. As a significant component element of smart city, college campus is gaining more attention with the concept of "Smart Campus" proposed. How to construct smart campus become a topical issue. In this work, we focus on three topics around smart campus: students interests mining, educational guiding and college resource optimizing. We innovatively propose a scheme supporting these three topics aiming to assist creating smart campus. A prototype called OnCampus is designed and implemented based on the scheme. The aforesaid three requirements of smart campus can be well contented by OnCampus.
\end{abstract}
%
\begin{keywords}
Smart city, Smart campus, Interests Mining, Educational guiding, College resource optimizing, Campus emotion.
\end{keywords}
%
\section{Introduction}
\label{sec:intro}

%The construction of smart campus under the background of smart city
Information technology has brought tremendous changes to our lifestyle and living environment, which reflecting in daily livelihood,  communication, public security, energy, water resource etc. Comprised with pervasive networks, advanced electronics, various sensors, urban is on the way of evolving to be smart city. Smart city is defined as "multiple sectors cooperating to achieve sustainable outcomes through the analysis of contextual real-time information shared among sector-specific information and operational technology systems"~\cite{szabo2013framework}. Many researches contribute to smart traffic system, smart medical system, smart food system to support smart city. As a special epitome of society, college campus is a significant component element of city. Recent years, the concept of "smart campus" is proposed.

%development of mobile deveces and liked by college student
As we all know, mobile device performs to be an ideal assistant platform to improve the efficiency of campus life~\cite{chen2012oncampus}, since it is portable, easily accessible, as well as it appear to be a part of the daily life of millions college students. Recent years, Mobile devices are gradually been widely applied in the campus scene. For instance, Mobile Virtual Campus (MVC)~\cite{tan2010collaborative} and T3G~\cite{chu2010two} are two useful mobile learning tools for college students. Meanwhile, it is noticed that location based service (LBS) have also matured and become more reliable which provide much improvement for mobile living experience~\cite{zhao2012mobimsg}. College students are always fascinated by these advanced technologies, especially the smart phones. Studies have shown that the number of smart phones within college campus accessing the Internet increases at an overwhelming speed. As a consequence, mobile platforms and its' location based service seem to be important factors when considering smart campus.

%We regard college campus as a domain after analysis and consideration of its characteristic
College campus can be summarized to be a scene: containing high quality infrastructure, well-organized, full of students personal information and education resource, being students' long-time living place. Under such condition, smart campus is desired to be like this: 1) More accurate context aware and ubiquitous access to network. 2) The resources are allocated efficiently. 3) The decisions are made smartly based on revealing the objective principles. 4) Students can easily establish or enhance friendships based on intersecting circles. As well as students' freedom of speech can be protected. However, there is a huge gap between this wonderful vision and the current progress of building smart campus. There have been much attempts towards smart campus that focus on physical campus infrastructure and virtual intelligent services. Although many campus contexts are taken into account. However, most work concentrate on physical equipments or functional applications. In our opinion, paralleled development of either physical or virtual campus services is no better than a effective interweaving to constructing smart campus. Thus, we innovatively propose a scheme supporting smart campus by utilizing some theoretical concept and academic approach, such as interests mining, recommendation technology and educational guiding model. Our objectives are making campus a smarter and more friendly place to live.

%needs and our solution
In this work, we mainly make contributions on three aspects. 1) Introduce interests mining models based on the context aware, locations, students profiles and relationships. Establish interests circles based on this.  2) Provide a platform on which users can share knowledge, post ideas, ask or answer questions. Managers can collect and analyze students posts, aimed at master "campus emotion" and provide appropriate educational guiding. 3) Devote to optimized the campus resources, such as used books, surplus products etc. 4) We design and developed a system called OnCampus based on mobile devices to support smart campus.

The rest of this paper will be developed as follows. Section II introduces the state of art of the research of smart campus, as well as researches about interests mining. Section III presents our motivation and requirements for smart campus, including personalized service, scientific guidance and resources optimizing. And the following section IV is to discuss the scheme we proposed towards constructing smart campus. With the knowledge of our scheme, section V design and implement a prototype for both client side on mobile phones and server side on PCs. Finally, section VI gives a conclusion about our work.

\section{Related Work}

On the way of constructing smart campus, more attentions are remained on the infrastructures and the related applications involving both learning and teaching as well as living guidance in order to promote personal experience.

Yim et at.~\cite{yim2014design} design and implement a VOD(video on demand) system, which is a smart campus guide android App that recognizes the structure in which a user is interested and displays useful information about the structure. Liu et al.~\cite{liu2014research} think smart campus to be an inevitable trend in the development of digital campus construction, they expound the concepts of smart campus based on the cloud computing and internet of things, discuss the problems should be noticed after elaborating intelligent application platforms. A smart campus application prototype based on layered Semantic Data Access(SDA) has been introduced by Boran et al.~\cite{boran2011smart}, with which integrating heterogeneous data. Wei Zhou et al.~\cite{zhou2014exploration} elaborates his exploration on smart campus in terms of the construction of security system structure, based on cloud computing, the security system structure is designed and described at six levels. Based on BYOD(Bring Your Own Device), a secure Wi-Fi Internet browsing architecture is proposed by Sangani et al.~\cite{sangani2013wi} to protect the campus web-browsing network.

Although practical in each individual area, these projects work in very limited ways, either drawing a map of the campus or presenting e-books. None of them present the colorful campus life as a versatile system to assist college students, offering efficiency and convenience.

Researcher major in interests mining conduct their research mainly on topic model on web text, like blog or Microblog. Nearly little research focus on interests mining in campus environment. Kuang et al.~\cite{kuang2010user} build an exact model for user interests by users�� log and users�� learning condition from an e-learning system so that can automatically identify the learner��s interests and recommend interest-related resources. Han et al.~\cite{han2014data} conduct their research on Web log mining and propose a data preprocessing method based on user characteristic of interests and show the superiority and recommended value of this new method.Chen et al.~\cite{huayue2012study} uses topic model to build up users�� interest model, with which they can provide resource recommendation for users in the system.

Although researchers is sparing no effort exploring the constructions of smart campus, of which the development seems to be still at the primary stage. Some researchers analyzed the characteristics and trend of smart campus, reminding the problems need to be paid attention in the process of constructing smart campus. The state of art of achievements seem to lie mainly in the construction of infrastructure and service application. Some other attempts on improving the security or efficiency of campus life via cloud computing and internet of things. These efforts benefit the users from a substantial but relatively restricted way like navigation service. Our work means to contribute to establishing smart campus, proposed a efficient scheme composed of interests mining, resources optimizing and scientific guiding in which we propose a new concept, campus emotion, to offer a better personal experience of campus life for college students.

\section{ Motivations and Requirements}

We aim to serve college students a versatile and talented campus experience, meanwhile giving a practical hand on campus life. What we are going to do is to provide a feasible scheme for satisfying their requirements, thus facilitating the contribution of smart campus which means an efficient and convenient college life.

The following scenario illustrates the requirements to be satisfied towards smart campus:

With great passion and curiosity, Bob entered the university dreaming a colourful college life. In campus environment, students, researchers and many other entities performed various activities. Facing the mass information of the multifarious social activities displayed, he felt at sea because he didn't know what he was really interested in. Under the urge of his great passion, he chosen some social activities and clubs, which he left and proven to be not accord with his interest after a period of time, because of the participation of the students surrounding him. The same embarrassment occurred when he chosen his profession one year later, but this time he could NOT leave. No longer did he have the passion or desire to try new things. He found it really difficult to conduct his exploration on his research without a cooperator.

One day, he took part in a demonstration against the school institution because of the invitation and complainants which he took for granted from his roommate who had decided to demonstrate for the same reason. When he graduated, he had to throw away some article for daily use or books which was useful and helpful for other students.

Bob's experience is the epitome of college experience of most university student currently. We survey students about their daily life to find out students' requirements. Actually, the requirements of university students are diverse and cover their study, living and entertainment. We can submit the requirements shown in Fig. \ref{Fig1:overview} when considering constructing smart campus:

\begin{figure}[htb]
\centering
\includegraphics [width=3in]{Fig1.pdf}
\caption{Overview of Requirements}
\label{Fig1:overview}
\end{figure}
\begin{enumerate}
  \item Personalized Service. It will be wiser and more beneficial for smart campus to provide information and service depending on the users' interest and profile. Otherwise the information explosion may dispel the enthusiasm of campus actors including students and teachers. Proper recommendation based on interests mining in campus environment, which will be discuss in next section, is a very good solution we proposed to offer personalized service.
  \item Scientific Guidance. The community of college students, as a particular group of people, are mostly comprised of enthusiastic youth. Though they are well educated, it is the long time education in school makes them lack of experience or well judgement. Meanwhile, they are eager to make contribution to family and society with great passion. All of these characteristics lead them easy to be cat's paw. In the next section, we will introduce our vision of conduct scientific guidance based on predicting the community's behavior and emotion. We also propose the phrase "campus emotion" vividly in our scheme.
  \item Resources Optimizing. The concept of resource in smart campus is not restricted in entity resources. Moreover, the information, context, event and even users themselves, can be particular resource which should be optimized. We are verifying a recommendation algorithm utilizing friendship of the user, which we regard as a solution to resources optimizing.
\end{enumerate}

\section{Proposed Scheme}

%Assumption
In our work, we make following 2 assumptions: 1) Pervasive access to network: we assume that every user such as students, faculties is equipped with a personal device(PC, smart mobile device, PDA etc). In smart campus systems, devices owned by users and the infrastructure form the condition that in campus situation\cite{khabou2014threshold}, users maintain a constant connection with each other, they are benefitted from a pervasive access to network, such as social network, learning network etc. 2) Precise context aware: it is supposed that the smart campus system has the powerful capability of context aware and pervasive computing\cite{khan2007future}. Besides the users' profile, the environment users involved such as building, location will be precisely obtained by advanced context extraction, identification and integration for smart campus.

The scheme we proposed is composed of interests mining, scientific guidance and resources optimizing based on the upper assumption:

As mentioned before, we put interests mining forward into smart campus to offer personalized service from a unique perspective. Different from the state of art interests mining mainly on semantic analysis from web or web behavior. We take campus context into consideration, from information of location, time, frequency, track and accompany, to infer the users' interests or needs. Then an customized recommendation strategy will filter the explosive information or resources to provide personalized service. For example, if a user appears in library frequently, always reading magazines about astronomy. Users with the similar interest may be recommended. This can be an effective way to find potential friends or cooperators. From another perspective, it is also a example of optimizing the resources in campus environment.

To conduct scientific guidance to students, smart campus should offer faculty the conditions of the students. Based on the analysis of most students context, combined with the technology of public opinions analysis. The prediction of the community behavior and emotion can be conducted. we pick up a new phrase called "campus emotion" to evaluate the state of most students. If the campus emotion is healthy or normal, it represents most students are in a positive mood, but if it presents sick, it means that most students are in negative mood. For example, we can infer the impactive of some students via the technology of public opinions analysis. A customized guidance scheme can be formulated by supervisor.

As we say, we pay attention to optimizing resources from various aspects. Besides mining potential resources to users, we are trying new recommendation algorithm to satisfy the needs of users. Recommendations cross different domain can be made under the consideration of friendship, this work is under tested during which seem to have a better performance than traditional algorithm.

We, towards smart campus, propose a scheme composed of interests mining, scientific guidance and resources optimizing corresponding to the requirement mentioned before. Based on the Assumptions, the scheme is a vision on the constructing smart campus. Moreover, we put forward some feasible attempts, some of which is under verifying to be efficient. Based on the scheme, we design and implement a prototype named "OnCampus", which will be elaborated in detail in the next section.
%-------------------------------------------------------------------------end------------------------------------------------------------------------------------%

\section{Prototype Implementation}
The prototype is given a name of "OnCampus" to proclaim our aim to serve college students a versatile and talented campus experience. Fig. \ref{Fig2:architecture} shows architecture of OnCampus. This proposed architecture enables the OnCampus to work as a practically versatile campus assistant, archiving real-time messaging and data transmission.  More than simple clients and servers, this architecture tempts to propose location-based interaction as well as rational communication mechanism. Servers appear a two-tier structure, providing service and storing information. The first tier server is a message sending server, storing the information uploaded by mobile clients and transferring requested data to clients�� functional modules. And the second tier is a location searching server which merely collects location information and acquires clients�� location data by interacting with location service module of clients. Clients include the interface of our OnCampus assistant which interacts with users. A client makes of three modules covering important college activities: circle communication, local trading and campus forum. With the introduction of location based service (LBS), OnCampus is favorable for students to establish and enhance friendships, knowledge exchange and affective interaction, buy and sell used goods in a campus flea market.

\begin{figure}[htb]
\centering
\includegraphics [width=3.5in]{Fig2.pdf}
\caption{Architecture of OnCampus}
\label{Fig2:architecture}
\end{figure}

\subsection{Communication Mechanism}
In the OnCampus architecture, clients communicate with message sending servers on service data and exchange location information with location search servers. This communication via XMPP protocol is quite frequent in order to archive real-time service and data refresh. Data response right coming after each refresh is not adequate concerning energy consumption and network flow. Thus practical communication mechanism is proposed to stipulate data transportation process between servers and clients. This section will further illustrate the structure of server side and discuss the communication process in detail.

\begin{figure}[htb]
\centering
\includegraphics [width=3.5in]{Fig3.pdf}
\caption{Communication process between client and two-tier servers}
\label{Fig3:communication}
\end{figure}

To achieve complicated location service at an energy-saving and efficient way, the OnCampus imposes two-tier server structure, LM2C (Location Search Server and Message Sending Server to Client), as shown in Fig. \ref{Fig3:communication}. And this two-tier structure involves message sending server (MSS) and location search server (LSS). Server side generally is responsible to collect data, classify and group the information automatically, and send messages according to various locations. And separately LSS is the server which manages physical area information and identifies related MSS, but MSS corresponds to store information and broadcast instant messages. Geological partitions are set by LSS, such as Teaching Building 1, Teaching Building 2 and Stadium M, and each physical area ID and its serving scope are documented as attributes of serving areas. With the database of serving areas, it is easy to check out that a certain physical location is assigned to a particular serving area, relating serving areas to a specified MSS. An MSS is in charge of several areas, and maintains a dynamical user list according to the users�� locations. MSSs accomplish the task to broadcast location-based instant messages, treating each physical area specifically.

On the basis of LM2C of server side, OnCampus architecture regulates the process of data transfer between servers and clients to support system service. The communication process of location-based instant messaging between clients and two-tier servers can be illustrated as Fig. 3. When a client connects with LSS, it submits location information to LSS. LSS then will feedback a set of data, including surrounding area attributes, identifiers of MSSs as well as the corresponding relations of connection permission. After receiving the data, client needs to check out which MSS it belongs to, and then client waits for permission from MSS.  MSS adds that client to broadcast list and deliver permission information to client. A message will only be pushed to the clients who are marked in the user list of specified MSS. When this client ranges out of the area of former MSS, the connection between client and former MSS will be dismissed and connection request will be asked to another MSS without communicates with LSS.

This two-tier LM2C architecture is designed to realize location-based instant message push, as well as energy efficiency. As mentioned above, client will receive a set of areas but not one attribute related to its exact location. This is designed for irredundant connection to LSS when location changes, clients decide the next MSS according to given datasets. Thus, an efficient way is implemented on the OnCampus system to use resource and bandwidth for servers. Besides, OnCampus supports that clients can passively receive messages rather than initiatively obtain messages, reducing energy consumption for clients

\subsection{OnCampus at a Glance}

OnCampus, whether its architecture with the use of location context or the functions modules designed in the clients, gives a potential condition and feasibility to conduct the scheme we mentioned before. Three function modules that are supposed to provide service involving learning, living and entertainment are elaborated in this section. They are the "Group" module, "Buy \& Sell" module and "Forum" module, all of which can make it feasible to conduct interests mining, scientific guidance and resources optimizing. For example, data from circle communication and local trading is conducive to mining the user's interest, thus recommendation can be customized. Moreover, scientific guidance can be formulated if we can infer the campus emotion reflected by the campus forum.

This prototype OnCampus is implemented on the android platform for the clients, and the server component is implemented in Java on either Windows OS or Linux OS.

\begin{figure}[htb]
\centering
\includegraphics [width=3.5in]{Fig4.pdf}
\caption{The prototype implementation of OnCampus }
\end{figure}

The "Group" module provides campus with functions of social network which is of great popularity among college student and meets the need of students�� social communication. Circles with various topics can be built up for students to communicate with the users who have similar interest. This helps students to make friends, improve their perception, and broaden their horizons. Personal friends can be added like some instant message systems, and status delivery, share, retweet among circles are supported by myCircle as well. For example, a user delivery a status that he/she is happy today, everyone within his/her circles will be able to view this, and his/her friends are the ones who have privilege to reply, share, or forward this status.

The "Buy \& Sell" module plays a practical role in campus life for students to exchange their goods. This module is designed to solve the problem of resource waste that students have to throw away their article for daily use or books which are useful and helpful for other students. With "Buy \& Sell", students can release the information about what are not among their necessities and are willing to sell to others. Customers thus will be able to find their needs and complete transactions with the information shown in this module. Users have the rights to view product lists, search certain products, release products offered for sale, and manage his/her on sale product list. Product lists are displayed according to various types of the products, so students can look over a certain type of products with an easy slide on type tabs. Usually the products in these lists will be set as for sale, whereas the product that has been sold can be either marked as fact or deleted from lists as its owner orders.

The "Forum" module is a practical implementation for knowledge sharing, affective interaction and Q \& A. This module is designed to provide a platform for students to share their knowledge, release some news, poured forth their feelings and ask for help. Any register members of OnCampus can read and reply the posts. The owner of the posts can delete, edit, or share his/her posts. They have full power to manage it. Everyone here enjoy full freedom of speech.

Convenience is pursued in university in many scenarios, and OnCampus can be an option at these circumstances. For example, Jack is interested in skiing sport very much and wants to find students who have similar interests. One solution is to look for circles with such topics in myCircle module in OnCampus, where holds expected students. And if no such circles can be found, Jack can build up a circle with skiing topic to attract students. Jack will find good friends who have the same hobbies with him in this circle. Another case is that with a new computer keyboard, students would rather prefer to sell the old and unbroken one on myFleaMarket to the student who needs it, in an economical and environmental friendly way.

\section{Conclusion}

In this paper, we realize the inevitable trend of smart campus. A feasible scheme composed of interests mining, scientific guidance and resources optimizing is proposed towards the construction of smart campus after analyzing the requirements to be satisfied. Beyond paralleled development on theory or entity, we design and implement a prototype proven to be efficient based on the proposed scheme.


% Below is an example of how to insert images. Delete the ``\vspace'' line,
% uncomment the preceding line ``\centerline...'' and replace ``imageX.ps''
% with a suitable PostScript file name.
% -------------------------------------------------------------------------
%


%\begin{figure}[t]
%\begin{minipage}[b]{1.0\linewidth}
%  \centering
%% \centerline{\epsfig{figure=image1.ps,width=8.5cm}}
%  \vspace{1.5cm}
%  \centerline{(a) Result 1}\medskip
%\end{minipage}
%%
%\begin{minipage}[b]{.48\linewidth}
%  \centering
%% \centerline{\epsfig{figure=image3.ps,width=4.0cm}}
%  \vspace{1.5cm}
%  \centerline{(b) Results 2}\medskip
%\end{minipage}
%\hfill
%\begin{minipage}[b]{0.48\linewidth}
%  \centering
%% \centerline{\epsfig{figure=image4.ps,width=4.0cm}}
%  \vspace{1.5cm}
%  \centerline{(c) Result 3}\medskip
%\end{minipage}
%%
%\caption{Example of placing a figure with experimental results.}
%\label{fig:res}
%\end{figure}
%
%\begin{eqnarray}
%y &=& ax^2+bx+c \nonumber \\
%~ &=& (x+p)(x+q)
%\end{eqnarray}
%shows an example of an equation layout.
%
%\begin{table}[t]
%\begin{center}
%\caption{Table caption} \label{tab:cap}
%\begin{tabular}{|c|c|c|}
%  \hline
%  % after \\: \hline or \cline{col1-col2} \cline{col3-col4} ...
%  Column One & Column Two & Column Three
%  \\
%  \hline
%  Cell 1 & Cell 2 & Cell 3 \\
%  Cell 4 & Cell 5 & Cell 6 \\
%  \hline
%\end{tabular}
%\end{center}
%\end{table}

% References
% -------------------------------------------------------------------------
\bibliographystyle{IEEEbib}
\bibliography{OnCampus}

\end{document}
